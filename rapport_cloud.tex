\documentclass[a4paper,12pt]{report}

\usepackage[utf8]{inputenc}
\usepackage[french]{babel}
\usepackage[T1]{fontenc}
\usepackage[scaled]{helvet} % police
\usepackage{lmodern}
\usepackage{layout}
\usepackage[top=2cm, bottom=2cm, left=3cm, right=2cm]{geometry}
\usepackage{setspace}
\usepackage{verbatim}
\usepackage{moreverb}
\usepackage{listings}
\usepackage{graphicx}
\usepackage{shorttoc}
\usepackage{glossaries}
\usepackage{xcolor}

\include{chapterStyle}

% Redéfinition de commandes
\renewcommand\thesection{\arabic{section}}
% \renewcommand\thechapter{\Roman{chapter}}
\renewcommand*\familydefault{\sfdefault} %% Only if the base font of the document is to be sans serif

\makeglossaries
\newglossaryentry{ASP}{
	name={ASP}, % apparait dans le glossaire
	text={ASP*}, % apparait dans le texte
	description={Application Service Provider,  Fournisseur d'Aplications de Service}
}

\title{Le cloud et la virtualisation}
\author{Sébastien Corbin et Gaëlle Avrillon}
\date{\today}


\begin{document}
% Interlignage 1,5
\begin{onehalfspace}

		\begin{titlepage}
			\begin{center}
				Sébastien CORBIN et Gaëlle AVRILLON\\
				CSII 3\ieme année\\
			\end{center}
			\hrulefill
			\vspace{7cm}
			\begin{center} 
				\LARGE \textbf{La virtualisation et le cloud}\\
			\end{center}
		\end{titlepage}
		\newpage

		\shorttableofcontents{Sommaire}{0}
		\setcounter{page}{1}
		\thispagestyle{empty}
		\newpage

		%%%%%%%%%%%%%%%%%%
		% Introduction 
		%%%%%%%%%%%%%%%%%%
	\chapter*{Introduction}
	
	\paragraph*{}
	Les Bureaux de Services (\emph{Service bureaus}) sont l’ancêtre du Software as a Service. Datant des années 60-70, leur point commun est qu’ils offrent un service en échange d’honoraires. Leur application dans les domaines est vaste : banques, assurances, etc. Une entreprise de ce type offre à ses clients son expertise dans un domaine : cela correspond tout à fait avec la notion actuelle de SaaS et c’est par l’avancée technologique des réseaux de communications que cette notion de “bureaux de service”  s’est énormément développée dans l’informatique.
	
	\paragraph*{}
	Les débuts du cloud computing se situent dans la notion de Fournisseur de Service d’Application (ASP*), présente au début des années 2000. Les premières applications à avoir migré dans les nuages sont les messageries, les outils collaboratifs, le CRM, les environnements de développement. Amazon, Yahoo et Google sont les premiers à se lancer dans ce concept. Ces deux derniers offrent au grand public des applications simples et gratuites telles que la messagerie ou la gestion de calendriers en ligne.
	
	\paragraph*{}
	La notion de Software as a Service (SaaS) est apparue en 2007 et remplace les précédents termes tels que ASP (Application Service Provider) ou encore “On Demand”.
	SaaS est un concept consistant à proposer un abonnement au logiciel plutôt que l’achat d’une licence. Avec le développement des technologies de l’information, de nombreuses offres SaaS se font à travers le web. Il n’y a alors plus besoin d’installer d’application de bureau, mais d’utiliser un programme client-serveur.
	Les applications s’appuyant sur ce modèle ont été nativement conçues pour le web contrairement aux modèles précédents.
	
	\paragraph*{}
	Le Grid Computing est une infrastructure virtuelle constituée d’un ensemble de ressources informatiques. Ces ressources sont partagées (elles sont mises à disposition des consommateurs), distribuées (elles sont situées dans des lieux géographiques différents), hétérogènes, coordonnées, autonomes et délocalisées (les ressources peuvent appartenir à plusieurs sites).
	
	\chapter{Cas d'entreprise}
	
	\chapter{Conclusion}
	
	% Table des matières
	\addcontentsline{toc}{chapter}{Tables des matières}
	\tableofcontents
	\newpage

 	% Interlignage 1,5
	\end{onehalfspace}
\end{document}